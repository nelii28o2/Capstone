\autsection{Team Aguacate Members Biographies}{Nelián Colón}
\label{sec:bios}

\subsection*{Nelián E. Colón Collazo}

Nelián E. Colón Collazo was raised in Orocovis, Puerto Rico. She is a senior
student currently pursuing her bachelor's degree in Computer Engineering at the
University of Puerto Rico at Mayagüez. Nelián has a big passion for programming,
reason why she is currently specializing in Software. She has taken courses like
Database Systems, Software Engineering, Structure and Properties of  Programming
Languages, Competition Programming and Analysis and  Design of Algorithms. She
has been participating in programming competitions  since she was in her second
year in the University, always arriving in the top  3 places. Although Nelián's
passion is Software, she also has experience  with Hardware. She took
Microprocessor Interfacing and Digital Systems Design.  She has been involved in
undergraduate research both in her University  and in the University of Texas at
El Paso working with CUDA and GPUs,  and has done two summer internships at
Honeywell Aerospace, in which she worked  with improvements to applications for
DM Gumstix CubeSat technology,  and Harris Corporation, in which she worked with
image coordinate conversion  using CUDA and C++, and range clipping algorithms
and a LiDAR images exploiter  using C\# and Kinect. Nelián also likes to be
involved in leadership positions.  She is currently the ACM-ECE's treasurer and
the Tau Beta Pi - PR Alpha's  Computer Engineering Representative, and has had
other positions in the past, such as President and Vice-President of the IEEE
Women In Engineering. Apart from being  a programmer, Nelián is also a musician.
She plays the Puerto Rican Cuatro and Tiple  with her well-known father, Edwin
Colón Zayas.

\subsection*{Samuel A. Rodríguez Martínez}

Samuel A. Rodríguez Martínez was born Chicago, IL, United States of America. He
moved to San Juan, Puerto Rico when he was 2 years old. While at San Juan he
developed his passion for software development, and enrolled for a Computer
Engineering B.S. degree at the University of Puerto Rico at Mayagüez. While at
the University, he competed in multiple programming competitions, both locally
at the campus, and in inter-university competitions. He placed 1st four times,
2nd one time and 3rd two times. He also competed in the worldwide IEEExtreme
competition and has placed first campus-wide. He was then involved in
undegraduate research at the University of Puerto Rico, where he worked with
GPGPUs for analyzing hyperspectral images. He was also given the opportunity to
perform an undegraduate research project at the University of California at
Berkeley where he wrote Python scripts for analyzing electronic voting logs and
created a website that hosted these tools for use by the general public. After
his research experience, he attended an internship at Google Inc. in Mountain
View, CA. During the internship he learned about GWT and Web front-end
development using Java. He excelled at his duties and was asked to join for
another internship. He attended his second internship during the summer of 2013,
where he learned more about front-end work with HTML5, CSS3 and Javascript.
During his whole student career, Samuel has worked with many technologies and
programming languages, including C, C++, Java, Javascript, Python, Perl, HTML5,
CSS3, SQL databases, Java Servlets, JSP, PHP, etc. Samuel is currently in his
final year at the university.

\subsection*{Daniel A. Santiago Rivera}

Daniel Santiago is currently a computer engineering student at the University of
Puerto Rico at Mayagüez. He is passionate about programming and interested in
mobile, backend, and web development. Daniel likes working with emerging
technologies such as Node.js and MongoDB. With his knowledge in these
technologies, Daniel has been offered to work in various places, including
Verizon Wireless, where he performed an internship during the summer of 2012. At
this internship, he excelled at his assignments by integrating an Android
application with a natural language engine, and as a result he was offered to
return for a second summer. But instead, he chose to work in a more challenging
environment at IBM for the summer of 2013, where he worked with a team of
interns producing new products with cutting edge technology like Node.js and
MongoDB. He wrote a backend service that outperformed the same service that was
written in Java by full time IBM employees. His work was cited in multiple
internal and external IBM publications. He has also been offered work on many
companies that need Android applications in a free-lancing manner, including
Microsoft. In his free time, after programming everything he plans for the day,
he enjoys cooking, running and assembling gaming computers.