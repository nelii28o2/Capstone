\autsection{Team Aguacate Members Biographies}{Nelián Colón}
\label{sec:bios}

\noindent{\textbf{\large{Nelián E. Colón Collazo}}}

Nelián was raised in Orocovis, Puerto Rico. She is a senior
student currently pursuing her bachelor's degree in Computer Engineering at the
University of Puerto Rico at Mayagüez. Nelián loves programming,
reason why she is currently specializing in Software. She
has been participating in programming competitions for 3 years, always arriving in the top 3 places. She has worked with CUDA in
undergraduate research in both in her University and in the University of Texas at
El Paso,  and summer internships at
Honeywell Aerospace, where she worked with applications for the DM Gumstix CubeSat technology, and Harris Corporation, where she worked with coordinate transform and range clipping using CUDA, and created a 3D image exploiter using C\# and Kinect. Nelián also likes to be
involved in leadership positions.  She is currently the ACM-ECE's treasurer and
the Tau Beta Pi - PR Alpha's Computer Engineering Representative, and has had
other positions in the past, such as President and Vice-President of the IEEE
WIE.

\noindent{\textbf{\large{Samuel A. Rodríguez Martínez}}}

Samuel was born Chicago, IL and soon moved to San Juan, Puerto Rico. While at
San Juan he developed his passion for software development, and enrolled for a
Computer Engineering B.S. degree at the University of Puerto Rico at Mayagüez
(UPRM) While at UPRM, he competed in multiple programming competitions, both
locally at the campus, and in inter-university competitions. He placed 1st four
times and 2nd one time. He also competed in the worldwide IEEExtreme competition
and has placed first campus-wide. He was then involved in undegraduate research
at UPRM, where he worked with GPGPUs for analyzing hyperspectral images. He then
performed an undegraduate research project at the University of California at
Berkeley where he wrote Python scripts for analyzing electronic voting logs and
created a website that hosted these tools for use by the general public. After
his research experience, he attended an internship at Google Inc. in Mountain
View, CA. He excelled at his duties and was asked to join for another
internship. He attended his second internship during the summer of 2013, where
he learned more about front-end work with HTML5, CSS3 and Javascript. During his
whole student career, Samuel has worked with many technologies and programming
languages, including C, C++, Java, Javascript, Python, Perl, HTML5, CSS3, SQL
databases, Java Servlets, JSP, PHP, etc.

\noindent{\textbf{\large{Daniel A. Santiago Rivera}}}

Daniel is currently a computer engineering student at the University of Puerto
Rico at Mayagüez. He is passionate about programming and interested in mobile,
backend, and web development. Daniel likes working with emerging technologies
such as Node.js and MongoDB. With his knowledge in these technologies, Daniel
has been offered to work in various places, including Verizon Wireless and IBM.
At Verizon Wireless he excelled at his assignments by integrating an Android
application with a natural language engine. At IBM, during the summer of 2013,
he worked with a team of interns producing new products with cutting edge
technology like Node.js and MongoDB. He wrote a backend service that
outperformed the same service that was written in Java by full time IBM
employees. His work was cited in multiple internal and external IBM
publications. He has also been offered work on many companies that need Android
applications in a free-lancing manner, including Microsoft. In his free time,
after programming everything he plans for the day, he enjoys cooking, running
and assembling gaming computers.
