\autsection{Impact}{Daniel Santiago}

This project aims to impact Educational Institutions, specifically Computer Science, Software Engineering
and Computer Engineering instructors that have to manually evaluate code submitted
by students. It will impact this community by providing an easy to use solution
that will save them time when evaluating student's coding assignments. In the
long-term, it will also increase the student's skills in programming areas by
giving recommendations on their codes referencing well proven standards and
patterns. Moreover by making instructors save time they can be more engaging in
other educational activities involving the students. From a different
perspective, this project will also decrease the chances of computer viruses being spread
trough the former evaluations methods.

\subsection{Commitments}
The system:
\begin{itemize}
\item Will have a database in a remote server.
\item Will have a web interface.
\item Can manage accounts (create, edit and delete).
\item Can receive code submissions.
\item Will support C++, Java and Javascript code.
\item Can evaluate submitted code.
\item Can present results of evaluations.
\item Will enable communication between the grader and the student through
inline comments.
\item Will display linter's results of submitted code in the web application.
\item Will automatically evaluate submitted code.
\item Will send emails to students and graders when code is evaluated.
\item Will be visually appealing.
\end{itemize}

\subsubsection{Limitations}
\begin{itemize}
\item The web application is not guaranteed to work properly on non supported
browsers.
\item It will not be able to process programming languages other than the
supported ones.
\item The web application will not work properly when no internet connection is
available.
\end{itemize}

\subsubsection{Legal Commitments}

The licenses for the tools that the team intends to use to build this system will be non
propagated (i.e not Copylefted), permissive open source licenses e.g. MIT
License.