\autsection{Impacts}{Daniel Santiago}

This project aims to impact Educational Institutions, specifically Computer
Science, Software Engineering and Computer Engineering instructors that have to
manually evaluate code submitted by students. It will impact this community by
providing a solution that will save them time when evaluating
student's coding assignments. In the long-term, it will also increase the
student's skills in programming areas by giving recommendations on their code;
referencing well-proven standards and patterns.

\subsection{Commitments}
\subsubsection*{Specifications}
The system will:
\begin{itemize}
\item Be able to manage accounts (create, edit and delete).
\item Be able to receive code submissions.
\item Support Java and Javascript code.
\item Evaluate submitted code.
\item Present results of evaluations.
\item Enable communication between the grader and the student through
inline comments.
\item Display linter's results of submitted code in the web application.
\item Send emails to students and graders when code is evaluated.
\end{itemize}

The tool creates reports of the code's quality that greatly helps the student
in their particular coding area and helps them follow good coding practices. It
will also include in-line commenting capabilities for enhancing the
communication between the graders and students, eliminating the need of slow
email exchanges.

\subsubsection*{Scope}

The system will contain the features above but will only evaluate code quality
to some extent. Specifically, time complexity analysis, implementation patterns,
logical naming of variables, relevancy of comments, object oriented structure
and code security will not be evaluated.

\subsubsection*{Deadlines}

The testing and quality framework will be completed by October 15, 2013. The
RESTful back end server that will communicate with the framework will be
finished by October 22, the accounts and repositories manager will be complete
by October 24, and the front end web client will be complete by October 25.
After all components are finalized, they will be integrated together by November
13 and finally deployed by November 15, 2013.

\subsubsection{Limitations}
\begin{itemize}
\item The web application is not guaranteed to work properly on non supported
browsers. A list of expected supported browsers can be found in Section~\ref{sec:stand}.
\item It will not be able to process programming languages other than the
supported ones.
\end{itemize}

\subsubsection{Legal Commitments}

The licenses for the tools that the team intends to use to build this system
will be non propagated (i.e not Copylefted), permissive open source licenses
e.g. MIT License.