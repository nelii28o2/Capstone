\part*{Executive Summary}

Assignments for Software Engineering classes have particular characteristics
that make them time-consuming to grade by professors and teaching assistants. In
the majority of cases, they are handed in electronically. The grader then runs
the code with some specific input to verify its correctness. Students might
choose to use different programming languages, different IDEs, and different
tests, making grading even more difficult. Additionally, many students write
code in different styles, which may make it harder to read for a person that has
been accustomed to a particular coding style. Furthermore, communication between
the grader and the student may be slow, often relying in sluggish e-mail
exchanges. All of this makes grading software assignments an enduring and
tedious task.

The main objective of this project is to design and build a Web-based
application  for eliminating the difficulties that come with grading software
assignments. The application will host repositories that contain students' code.
When a student uploads code for an assignment, the system notifies the grader
and automatically runs test cases previously specified by the grader. The
results of the test cases are sent to the grader. The grader also has the option
to add line-by-line comments to the student's code, and run the application's
linter tools to verify code style. Finally, the grader will have the option to
assign a grade to the student.

The Web application will be fully hosted in the cloud. The goal with this
application is to greatly reduce the time that it takes for professors and
teaching assistants to grade assignments, and to improve the communication
between students and graders. The web application will also promote good coding
practices by giving feedback to the students when they submit their code.

The end goal of this project is to deploy this application in the cloud by
November 15. This will be done by first completing the testing and quality
framework by October 15. The RESTful back end server that will communicate
with the framework will be finished by October 22, the accounts and repositories
manager will be complete by October 24, and the front end web client will be
complete by October 25. After all components are finalized, they will be
integrated together by November 13 and finally deployed by our expected end
date.

% ROI = return of investment.

The total cost for the project will be \$41,863.87. The total cost is composed
of \$12,548.07 of personnel salaries, \$14,400 of employee fringe benefits, and
\$14,915.80 of overhead hours, which includes work like proposal writing,
meeting attendance, and presentations. There will also be additional recurrent
costs for hosting infrastructure after the Web application is deployed in
Amazon Web Services EC2, including the actual infrastructure costs which is
estimated to be \$528 annually and the cost of maintaining an SSL certificate,
which is \$400 annually. The total recurrent costs after deployment are \$928.
Therefore, it should not be difficult to make the service profitable with
Google ads.

% TODO(samus250): Fix the previous paragraph, since the last statement is kinda
% bogus. There is no evidence right now to support what is written.
