\part*{Executive Summary}

Assignments for Software Engineering classes have particular characteristics
that make them harder to grade by professors and teaching assistants. In the
majority of cases, they must be handed in electronically. The grader then runs
the code with some specific input to verify its correctness. Students might
choose to use different programming languages, different IDEs, and different
tests, making it even more difficult for grading. Additionally, many students
write code in different styles, which makes it harder to read for a person
that has been accustomed to a particular coding style. Furthermore,
communication between the grader and the student is very poor, often relying
in slow e-mail exchanges.  All of this makes grading software assignments an
enduring and irritating task.

This product is designed to eliminate the difficulties that come with grading
software assignments. It is a web-based application that will host
repositories that contain students' code. When a student uploads code for an
assignment, the system notifies the grader and automatically runs test cases
that were specified by the grader. The results of the test cases are sent to
the grader. The grader also has the option to add line-by-line comments to the
student's code, and run the included linter programs to verify code quality
and style. Finally, the grader will have the option to assign a grade to the
student.

The deliverable for this project will be a fully hosted web application. The
goal with this application is to greatly reduce the time that it takes for
professors and teaching assistants to grade assignments, and to help
communication between students and graders. The web application will also
promote good coding practices by giving feedback to the students when they
submit their code.

The development milestones for this project will be the finalization of the
front end web client, a complete RESTful back end server, the completion of
the accounts and repositories manager and the culmination of the testing and
quality framework. The end goal of the team is to integrate all of these
components and finally deploy this service in the cloud.

% ROI = return of investment.

The total cost for the project will be \$50,757.26 of development costs and
and additional scalable and recurrent cost for hosting infrastructure in
Amazon Web Services EC2. Expected profitability for this product is very high.
At only \$10,000 per educational institution license, the return of investment
(ROI) will be nearly balanced to \$0 at 5 licenses sold, but the team expects
to sell at least 25 educational licenses.
