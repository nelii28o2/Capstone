\autsection{Project Antecedents}{Nelián Colón}

\subsection{Previous Work Experience}

The members of Team Aguacate have known each other for several years, many of
which have been spent developing project for several courses and extracurricular
activities. They have participated together in programming competitions,
hackathons and projects, such as the Microprocessor Interfacing course project
on the sprint semester of 2013, WaveSphere \cite{Micro2}. These previous
experiences have allowed the team to build their teamwork abilities.

Since there are several online judges for programming competitions, such as UVa
\cite{UVA} and SPOJ \cite{SPOJ}, and since they have been through the experience
of having to submit code for programming labs, the team came up with the idea of
using something similar to an online judge to facilitate not only their lives by
making a tool to turn in programming assignments easier and getting grades
faster, but also the instructors' lives that have to grade hundreds of these
assignments.

\subsection{Similar Projects}

%We need more advantage-disadvantage and comparison to the proposed product

Educational Institutions outside of Puerto Rico have similar systems to the one
being proposed. In 1994, University of Maryland presented Kassandra: The
Automatic Grading System \cite{Matt1994}. As the title suggests, Kassandra is an
automatic grading system that is used for grading assignments in scientific
computing. This system is used by students to check the correctness of his/her
program assignments. This is achieved by comparing the program output with an
expected result. In 2000 the Virginia Polytechnic Institute and the Microsoft
Research Corporation developed Curator, a web-based electronic submission that
supports automated grading of elementary programming assignments. \cite{Curator}
Its grading mechanism is similar to Kassandra, the programs are compiled and
their output result is tested against with an specific input. A few years ago, in
the University of Puerto Rico at Mayagüez, a TA for the Data Structures course
and former student, José Santuche, developed a small script that would checkout
his students' code, run some tests, compare the output and send an email to the
student with the results. However it was unreliable and had to be manually
changed between assignments. The most currently competitive product is another
web-based tool named Web-CAT. Web-CAT is an open-sourced automated grading
system that is used to grade students on how well they test their own code,
instead of comparing output results. Web-CAT needs to be installed on an
institutions own infrastructure \cite{WebCat}.

\subsection{Our Approach}

From the aforementioned projects, Panda Code Review (PCR) integrates correctness
checking and privacy awareness. PCR differs from the previous efforts because it
is fully web-based and hosted in the cloud out of the box, so there is no
installation needed and the update process is transparent to institutions. Team
Aguacate aims to provide a complete grading tool that, in addition of output
correctness checking, it will include code quality checking, in-line commenting
capabilities for the grader to put comments if needed, an easy way to upload
code, among other distinctive features.

\subsection{Project Importance}

As stated by previous and current TAs from our University, such as José
Santuche, it is important to have an automatic grading tool because they become
exhausted of having to grade the same assignment over and over again, it is a
slow and irritating process, especially when the graders have hundreds of
students. The graders begin by verifying at all the details, but at some point
they just verify the output without looking at the code. From the students
perspective, the process of submitting code and waiting for a grade is slow and
annoying, too. PCR aims to tackle these problems by providing a centralized and
fully integrated web-based system in which students will be able to submit their
assignments and it will automatically grade them based on code quality and test
cases submitted by the TA or professor. Our project will incorporate the use of
repositories, a tool that is not much used in our University, so that it also
helps students to learn how to use version control and source code management
technologies.

\subsection{Standards and Regulations}

Since our tool will manage student data, it needs to be secure and private. It
also needs to maintain the test cases uploaded by the instructor private to the
students.

%TODO mention technologies standards aqui.
% Some standards that we can mention (ideas):
% HTTPS encryption for web site traffic
% Infrastructure standards (like RAID for backup... this might all be amazon
% EC2 specific)
% Have no idea of what else...
