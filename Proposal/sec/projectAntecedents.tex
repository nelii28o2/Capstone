\autsection{Project Antecedents}{Nelián Colón}

\subsection{Previous Work Experience}

The members have participated together in programming
competitions, such as IEEE Xtreme, Inter-Universitary Competitions at UPR Bayamón and UPR Ponce, and Java Battles at UPR Mayagüez, arriving at the top 3 places in most of them. They have also participated in hackatons, together and separated, such as the Google 24 Hours Of Good (Nelián) and UPRM's Hackaton (Daniel and Samuel). In addition, they have done some projects together, such as the Microprocessor Interfacing
course project on the spring semester of 2013, which they called ``WaveSphere"
\cite{Micro2}. These previous experiences have allowed the team to build their
technical skills and teamwork abilities.

The team's participation in programming competitions have exposed them to
several  online judges for programming competitions, such as Universidad de
Valladolid's (UVa) judge, \cite{UVA} and the Sphere Online Judge (SPOJ)
\cite{SPOJ}. Team Aguacate has also had the experience of submitting code
assignments for programming labs. These previous experiences have lead the team
to come up with the idea of creating something similar to an online judge to
facilitate not only their own work by making a tool to turn in programming
assignments easier and getting grades faster, but also the instructors' work
since they have to grade hundreds of these assignments.

\subsection{Similar Projects}

Educational Institutions outside of Puerto Rico have similar systems to the one
being proposed. In 1994, the University of Maryland presented Kassandra: The
Automatic Grading System \cite{Matt1994}. As the title suggests, Kassandra is an
automatic grading system that is used for grading assignments in scientific
computing. This system is used by students to check the correctness of their
program assignments. This is achieved by comparing the program output with an
expected result, much like current online judges do. In 2000, the Virginia
Polytechnic Institute and the Microsoft Research Corporation developed Curator,
a web-based electronic submission system that supports automated grading of
elementary programming assignments \cite{Curator}. Its grading mechanism is
similar to Kassandra; the programs are compiled and their output is tested
against a known correct output from a specific input. A few years ago in the
University of Puerto Rico at Mayagüez (UPRM), a former student and TA for the
Data  Structures course, José Santuche \cite{Santuche}, developed a simple
script that would check-out his students' code, run some tests, compare the
output and send an email to the student with the results. However, it was
unreliable and had to be changed manually between assignments. Currently, the
most competitive product is another web-based tool named Web-CAT. Web-CAT is an
open source automated grading system that is used to grade students on how well
they test their own code. This is different than just comparing output results.
Web-CAT needs to be installed on an institution's own infrastructure
\cite{WebCat}. A similar product to Web-CAT is Praktomat \cite{Praktomat}, an
open source web-based tool for automatic grading. This product differs in that
it also enables peer- to-peer reviews, but does not have a sophisticated testing
framework.

\subsection{Proposed Approach}

From the aforementioned projects, Panda Code Reviews (Panda) integrates
correctness checking and privacy awareness. Panda differs from the previous
efforts because it is fully web-based and hosted in the cloud out of the box, so
there is no installation needed and the update process is completely transparent
to institutions and users. Competing products like Web-CAT, Praktomat, Curator,
and Kassandra require additional infrastructure and a painful set up process.
They all have the hassle of having to manually back up the database and
distressingly updating the infrastructure's software. Moreover, Panda provides a
key missing feature in all other similar products: the code quality analysis.
This means that the tool creates reports of the code's quality that greatly
helps the students excel in their parcicular coding area and helps them follow
the best coding practices. It will also include in-line commenting capabilities
for enhancing the communication between the graders and students, eliminating
the need of slow email exchanges.

\subsection{Project Importance}

As stated by previous and current TAs, such as José Santuche, it is important to
have an automatic grading tool because the graders gradually become exhausted of
having to grade the same assignment over and over again; it is a slow and
irritating process, especially when the graders have hundreds of students. As
Jesús E. Luzón mentioned (quoted above), with only 23 students, grading a single
assignment might take him up to 3 hours of his time. As Santuche mentions:
\begin{quote} ``Since the process of grading is so exhaustive, the graders
normally begin by verifying all the details, but at some point they just verify
the output without looking at the code." \end{quote}. From the students'
perspective, the process of submitting code and waiting for a grade is slow and
irritating as well. The current way of submitting code is prone to mistakes when
submitting the assignments via email, and is prone to virus infections when
submitting the assignment via some portable storage device. Panda aims to tackle
these problems by providing a centralized and fully integrated web-based system
in which students will be able to submit their assignments and it will
automatically grade them based on code quality and test cases submitted by the
TA or professor. The automatic grading will not be applied immediately to the
student, since the professor could override the system and use the code quality
tools to determine a grade for the student. This tools helps the grader as much
as possible, without removing power. This project will incorporate the use of
repositories, a tool that is not much used in UPRM, so it also helps students
learn how to use version control and source code management technologies. All of
these solutions that the tool will provide makes it a great marketable product.
Institutions can greatly benefit by purchasing this product as a service.

\subsection{Standards and Regulations}

Since the tool will manage student data, it needs to be secure and private. It
also needs to maintain the test cases uploaded by the instructor private to the
students. To maintain the information secure and not visible by anybody, an
HTTPS connection will always be required between Panda and the client's
computers. All information will be encrypted in the system, including credential
information, account data, and repository data. Also, a very sophisticated,
third party cloud serving infrastructure will be used to guarantee data backup
and up to 99.99\% uptime: ensuring that the system will be available always when
needed.
