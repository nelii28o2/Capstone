\autsection{Project Antecedents}{Nelián Colón}

\subsection{Previous Work Experience}
The members of Team Aguacate have known each other for several years, many of which have been spent developing project for several courses and extracurricular activities, such as programming competitions.  These previous experiences have allowed the team to build their teamwork abilities.

Since there are several online judges for programming competitions, and since they have been through the experience of having to submit code for programming labs, the team came up with the idea of using something similar to an online judge to facilitate not only their lifes by making an easier tool to turn in programming assignments and getting grades faster, but also the instructor's that have to grade hundreds of these.

\subsection{Similar Projects}

In 1994, University of Maryland presented Kassandra: The Automatic Grading System \cite{Matt1994}.  As the title suggests, Kassandra is an automatic grading system that is used for grading assignments in scientific computing.  This system is used by students to check the correctness of his/her program assignments.  This is achieved by comparing the program output with the expected result.

A few years ago, a teaching asssitance (TA) for the Data Structures course and former student at our University, José Santuche, developed a small script that would checkout his students' code, run some tests and compare the results.

Panda Code Review (PCR) differs from the previous efforts because it is web based instead of a desktop application. PCR aims to provide a complete grading tool that, in addition of correctness checking, it will include code quality checking, in-line commenting capabilities for the TA to put comments if needed, an easy way to upload code, among other distinctive features.