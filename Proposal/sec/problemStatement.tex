\autsection{Problem Statement}{Samuel Rodríguez} %The problem explained

Most educational institutions manually grade programming (code) assignments
completed by students. Manual evaluation of code is inefficient: it takes time
and is prone to bad evaluations due to the substantial amount of different ways
a program can be written. This leads to only evaluate the program's results,
i.e. its output, based on an specific input. However such method of evaluation
can be maliciously exploited and does not measure code quality at all. Missing
the evaluation of such areas can lead professors to pass student with poor
coding performance. Currently, most professors and teaching assistants evaluate
output by manually importing code into a workstation, most probably the grader's
own personal machine, and then manually executing the program through some test
cases. This is also highly insecure because the code is not properly sandboxed*
(Contained in an environment). The code could be malicious and cause damage to
the running workstation or even worse get a hold of sensitive data.

% Quotes from TAs
Zach Cross a TA from the University of North Carolina, Chapel Hill comments when
he is asked if he has a grading tool for students code:

\textit{"Nothing well integrated and highly functional, mostly hacky bash/python
scripts..."}

Moreover he comments on the current system his institution uses for grading:

\textit{"We don't have a clean web interface or anything and they aren't
functional in the sense that they break for most assignments as soon as errors
occur e.g. they can grade the 100\%'s but not much else."}

%Here we hit stakeholders, how the project solves the problem and the scope of
%such solution

The proposed solution tackles each of the problems specified and also adds extra
value by integrating features that can help increase the student programming
skills. This product will directly affect the Professors, Teacher Assistants
(TAs) and the students. It will also indirectly affect the educational
institution that uses it. The project will create a solution that is secure,
reliable, and easy to use. The solution will not only help in the grading
process, but will also give helpful feedback that promotes better coding
practices on the submitted code.
