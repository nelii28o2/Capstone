\autsection{Problem Statement}{Samuel Rodríguez} %The problem explained

%TODO Anadir numeros (para concretizar las mejoras de tiempo que tiene el
% proyecto).

Most educational institutions manually grade coding assignments completed by
students. Manual evaluation of code is inefficient: it takes a great deal of
time and is prone to bad evaluations due to the substantial amount of
different ways a program can be written. This leads the grader to only
evaluate the program's output based on an specific input. Such a method of
evaluation can be maliciously exploited and does not measure code quality at
all. Missing the evaluation of code quality can lead professors to pass
students with mediocre coding skills. Currently, most professors and teaching
assistants evaluate output by manually importing code into a workstation, most
probably the grader's own personal machine, and then manually executing the
program through some test cases. This is highly insecure because the code is
not properly sandboxed \footnote{In computer security, a sandbox is a security
mechanism for separating running programs. It is often used to execute
untested code, or untrusted programs.}. The code could be malicious and cause
damage to the running workstation or could even get a hold of very sensitive
data.

% Quotes from TAs  

Zach Cross \cite{Zach}, a Teaching Assistant (TA) from the University of North
Carolina at Chapel Hill, comments when he is asked if he has a grading tool
for students code: \begin{quote} ``Nothing well integrated and highly
functional, mostly hacky bash/python scripts..." \end{quote} Moreover, he
comments on the current system his institution uses for grading: \begin{quote}
``We don't have a clean web interface or anything and they aren't functional
in the sense that they break for most assignments as soon as errors occur e.g.
they can grade the 100\%'s but not much else." \end{quote}

Jesús E. Luzón \cite{Chiki}, a TA from the University of Puerto Rico at
Mayagüez, comments when asked about the time it takes to grade a 50 line
coding assignment: \begin{quote} ``The way I grade the assignments is by
creating a testing project that only includes the student's code. I check out
the project from the students repository and then run the code and compare the
output. For a well written assignment, It takes me about 3 to 4 minutes to
evaluate it. But for an assignment that has errors, it takes me about double
the time: from 6 to 8 minutes. This happens because I need to figure out where
exactly the error comes from, so that I can give the student some useful
feedback." \end{quote} When asked about how many students he grades, he
comments: \begin{quote} ``I have 23 students, which doesn't seem like too
many, but the time it takes to grade their assignments adds up quickly. In the
worst case, I can spend 3 hours grading a single assignment." \end{quote}

%Here we hit stakeholders, how the project solves the problem and the scope of
%such solution

The proposed solution makes the grading of coding assignments more efficient by
providing an easy to use web-based application that handles submissions and
automatic grading of coding assignments. It is expected that the solution will
reduce the time it takes to evaluate a 50 line coding assignment by 75\%, from a
maximum of 8 minutes to a maximum of just 2 minutes. It will also completely
eliminate all overhead preparation efforts like setting up individual projects
and checking-in many different code repositories. It will not only grade based
on the program's output but also on code quality and performance. This solves
the problem of inaccurate grading by taking into account other factors that
measure a student's academic achievements. The proposed solution tackles each of
the problems specified and also adds extra value by integrating features that
can help increase the student's programming skills, e.g. it provides helpful
feedback that promotes better coding practices on the student's submitted code,
and it promotes usage of version control systems. This product will directly
benefit the Professors, Teaching Assistants (TAs) and the students. It will also
indirectly benefit the educational institution that uses it. The proposed
solution is secure, reliable, and easy to use.
