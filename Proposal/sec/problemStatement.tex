\autsection{Problem Statement}{Samuel Rodríguez} Most educational institutions
manually grade programming (code) assignments done by students. Manual
evaluation of code is inefficient, it takes time and is prone to bad
evaluations due to the big amount of different ways a program can be written.
This then leads to only evaluate the programs results, ie. its output base on
an specific input. However such method of evaluation can be tricked and does
not measure code quality. Missing the evaluation of such criterias can lead to
passing student with poor implementation ideas. The way the evaluation of
output is done is by manually importing code into a workstation, possibly the
owns TA personal machine and the manually executing the program trough some
test cases. This is highly insecure because the code is not sandboxed*
(Contained in an environment). The code can contain malicious code that can
eventually cause damage to the running workstation or even worse get a hold of
sensitive data.

The proposed solutions tackles each of the problems specified and also adds
extra value by integrating features that can help increase the student
programming skills. This will product will directly affect the Teacher
Assistants (TAs) and the student, it will also indirectly affect the
educational institution that uses it. The project will create a solution that
is easy to use, secure and reliable. It will help in the grading process and
will also give feedback on the submitted that if read and understood can lead
to better coding practices.
