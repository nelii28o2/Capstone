\autsection{Market Overview}{Daniel Santiago}

Our intention is to market this product as a service. We intend to charge
educational institutions a monthly rate based on the amount of students they
intend to support. If the institution desires to integrate the product with
their own infrastructure they can purchase the whole product, the institution
will receive the full open source product which they can modify and integrate
with any other service they desire to be used solely with the institution
students.

Income will come from subscriptions and these will be used to pay for hosting
infrastructure to support the cloud service, it will also pay for any further
work done by developers to update and maintain the cloud service.

\subsection{Potential Customers}

The service/product we intend to develop is aimed to educational institutions
that require evaluation of students developed programs in courses. The service
can also be used by interviewers ? Online Judges ? Small Companies?

\subsection{Current or Potential Competition}

There are some closed code review systems available in various university
institutions like \cite{Matt1994} University of Maryland's Kassandra. There are
also many online judge systems like UVa \cite{UVA} and SPOJ \cite{SPOJ}.
Additionally, many code linters and style guides exist. But all these services
are not combined into one.

\subsection{Competitive Advantages}

Our main competitive advantage is that we provide a service that is cloud and
web-based. It is reliable, secure and easy to use. It can throughly analyze
code, not just by comparing its result but by reviewing coding styles.




