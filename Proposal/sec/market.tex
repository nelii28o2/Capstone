\autsection{Market Overview}{Daniel Santiago}

% This quote can be better used here saying that the market for such products is
As the humans progress more processes are being automated, programming
assignments grading is one of these processes. The automation is done by
currently developed tools, however as we previously mentioned not many
educational institutions are using them. This means the market is open for
innovation and good competition. Zach Cross \cite{Zach}, a TA from the
University of North Carolina at Chapel Hill, comments when he is asked if he has
a grading tool for student's code: \begin{quote} ``Nothing well integrated and
highly functional, mostly hacky bash/python scripts..." \end{quote} Moreover, he
comments on the current system his institution uses for grading: \begin{quote}
``We don't have a clean web interface or anything and they aren't functional in
the sense that they break for most assignments as soon as errors occur e.g. they
can grade the 100\%'s but not much else." \end{quote}

From a market perspective, the proposed solution is a service, because there
will not be a physical product we can deliver, instead users will use the
solution through the Internet. Competing products are open-source which has lead
us to use a free of charge with revenue in advertisements business model. Income
from ads will be used to pay for hosting infrastructure to support the cloud
service, including the SSL certificate. It will also pay for any further work
done by developers to update and maintain the cloud service. The business model
is reasonable because we can easily serve the right ads due to the knowledge of
the targeted audience that will be using the service.

\subsection{Potential Customers}

The product that the team intends to develop is aimed to educational
institutions that require evaluation of student's coding assignments. The
service can potentially be used by interviewers and peer to peer collaboration
in a company.

\subsection{Current or Potential Competition}

There are some closed code review systems available in various university
institutions like University of Maryland's Kassandra \cite{Matt1994}. There are
some other open source products available as well, like Web-Cat \cite{WebCat}
and Praktomat \cite{Praktomat}. There are also many online judge systems like
UVa \cite{UVA} and SPOJ \cite{SPOJ}. Additionally, many code linters and style
guides exist. Some of these products are outdates (Kassanda, 1994), other
required dedicated infraestructure and installation (WebCat and Praktomat) and
some others are not targeted to educational institutions (UVA SPOJ). Moreover
none of these competitive products include code quality analyzers.

\subsection{Competitive Advantages}

The main competitive advantage of this product is that it provides a service
that is cloud and web-based. It can analyze code, not just by comparing its
result but by reviewing coding styles. Additionally, this system does not
require any installation at all.
