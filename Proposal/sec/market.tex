\autsection{Market Overview}{Daniel Santiago}

The intention of the team is to market this product as a service. It is intended
to give the service free of charge.

Income will come from advertisements and these will be used to pay for hosting
infrastructure to support the cloud service, including the SSL certificate. It
will also pay for any further work done by developers to update and maintain the
cloud service.

\subsection{Potential Customers}

The product that the team intends to develop is aimed to educational
institutions that require evaluation of student's coding assignments. The
service can potentially be used by interviewers and peer to peer collaboration
in a company.

\subsection{Current or Potential Competition}

There are some closed code review systems available in various university
institutions like University of Maryland's Kassandra \cite{Matt1994}. There are
some open source products available as well, like Web-Cat \cite{WebCat} and
Praktomat \cite{Praktomat}. There are also many online judge systems like UVa
\cite{UVA} and SPOJ \cite{SPOJ}. Additionally, many code linters and style
guides exist. All of these competing products are outdated
%TODO decir porque decimos esto
; they all require
installation on an institution's own infrastructure, and they don't include any
code quality analyzers at all.

\subsection{Competitive Advantages}

The main competitive advantage of this product is that it provides a service
that is cloud and web-based. It is reliable, secure and easy to use. It can
throughly analyze code, not just by comparing its result but by reviewing coding
styles. Additionally, this system does not require any installation at all.
