\autsection{Technical Plan}{Samuel Rodríguez and Daniel Santiago}

\subsection{Web Application}

Panda Code Reviews is a fully hosted web application. The web application
includes many parts, the main of these are the front end and the back end. The
front end is what the user sees and interacts with directly. It is very
important to design the front end with ease of use in mind, and almost equally
important is designing the interface with a clean and modern look. The back end
is the underlying software that manages the communication between the front end
interface and the database, repository manager and testing framework. The front
end and the back end are connected in various parts, so this must be properly
designed for ensuring compatibility and hassle-free communication. For example,
data passing between the database and the front end through the back end must be
concise, so a standard object model should be used. It is discussed in the
following sections the usage of JSON objects for this purpose. Additionally, the
URL Endpoints of the application must be specified for compatibility between the
front end and the back end. These are specified in later sub sections.

\subsection{Front end}

As mentioned earlier, it is very important to design the front end interface
with ease of use in mind, and it is also very important to design the interface
with a clean and modern look. The Model-View-Controller (MVC) pattern is being
followed for developing the front end. The idea of the MVC pattern is to have a
good organization in the code between managing the application's data (model),
the application's logic (controller), and the application's presentation (view).
The view gets data from the model to display to the user. When a user interacts
with the application, the controller changes data in the model, and the data
gets passed from the model to the view so that it is displayed to the user. In
the case of Panda Code Reviews, the view of the application is the Document
Object Model (DOM) of the web pages. The controller of the application is
JavaScript code that uses AngularJS, and the model data is all stored in JSON
objects.

One of the most difficult tasks in developing the front end of the application
is ensuring good compatibility between the model, the view and the controller.
AngularJS makes this task easier by providing a templating engine in the client
side. It allows for automatic refreshing of data between the model, view and
controller. For example, if the controller of the application changes the model
data, AngularJS automatically refreshes the view so that it gets displayed to
the user. And if the view is changed by the user, AngularJS automatically
updates the model.

\subsubsection{View}

The view is being handled by HTML and CSS. Panda Code Reviews uses Bootstrap, an
open source front end framework for faster and easier development of the front
end's view. It provides many CSS classes for easily templating the web pages. It
is also highly customizable, and provides a JavaScript plugin interface for
enabling animations and other features like component pinning and affixes. These
plugins might be activated by the controller or by other means such as after a
page loads in the client's browser.

\subsubsection{Controller}

The controller of the front end is being developed in JavaScript. JavaScript is
a scripting language implemented by all major browsers and provides a very
extensive standard library for manipulating elements of the DOM and registering
event listeners. JQuery, a comprehensive JavaScript library, is being used for
simplifying and reusing many routines. Additionally, AngularJS is also embedded
in the controller. It is explained in the next section.

\subsubsection{Model and Communication with the Controller}

The model of the application is embedded in the controller and specified in the
database, which the front end has no direct access to. Instead, the front end
will communicate with the back end through a REST API, and the back end will
answer the front end's query with JSON objects. AngularJS will be used for
simplifying the communications between the front end's controller and model.
AngularJS will also be used for communicating with the back end through its REST
API. When the user interacts with the front end's view, AngularJS responds by
changing the model in the front end, which will then call the necessary
controller functions that will communicate with the back end through its REST
API. The back end will respond to the controller's query with JSON objects, and
AngularJS will manage the change of data between the controller, the model and
the view. AngularJS is also highly modifiable, so functionality can be
overridden so that it better suits Panda Code Reviews's functionalities.

\subsubsection{Views}

As the time of this writing, the team has developed some static web pages with
HTML4/5 and Bootstrap CSS. This doubles as part of the implementation of the
front end and as mockup creation for the design phase. The application views
that have been planned by the team follows. Use case diagrams can be found in
Appendix~\ref{sec:useCase} and descriptions can be found in
Appendix~\ref{sec:useCases}. Screenshots of what the users is expected to see
are found in Appendix~\ref{sec:mockups}. Sequential diagrams that describe the
actions that the users can perform can be found in Appendix~\ref{sec:seqs}.

\begin{enumerate}

\item \textbf{Site Home Page:} The site homepage will include information and
marketing information about Panda Code Reviews and the services that it offers.
It will also include links for logging in and signing up. If the user is already
logged in, it will also include links for accessing his account home page and
general account settings.

\item \textbf{Login and Sign Up:} The login screen and sign up screen asks for
the users e-mail and password. If the user clicks the sign up button, it will
also ask for the user's name and for a password retype.

\item \textbf{Account Home:} The account home will show different things
depending on the user. If the user is a student, the account home will show the
enrolled courses and will also contain a button for enrolling new courses. Each
course will also contain a snippet of the assignments that are due and the dates
at which they are due. The student could click on the course name to see more
information about the course, and he could also click on the due assignment
names to see more information about the assignments. In the case of a professor,
he would be able to see the list of courses, and could also create a new course.
The professor would also be able to see the amount of students enrolled in the
course and the names, contact emails, and summary of grades of each student.

\item \textbf{Create Course:} The professor is the only one that can see the
create course view. This view is included as part of the account home of the
professor, it is a drop down that asks for the new course's name. After creating
the course, the professor can edit it and add students.

\item \textbf{View Course:} Students and professors can view the courses. in
case of the student, he will be able to see basic information about the course,
like course title, number, semester, and year. He can also see a summary of his
grade, his assignments, and his submissions. He can also see contact information
of his professor in a glance. A professor can see with great detail the amount
of students in the course, and the assignments that he currently has in the
course. He can also edit and delete the course from this view, and can create
new assignments to add to the course.

\item \textbf{Enroll Course:} The student can easily enroll in a course in the
enroll course view which is part of the student's account home view. This view
is a drop down that only asks for the courses name. This is a searching feature.
As the student types up the name, it will instantly provide results. The student
can then select a course and enroll in it.

\item \textbf{Create Assignment:} The professor can create an assignment in his
course view. This view is a drop down, where the professor can add the name and
the due date of the assignment.

\item \textbf{Assignment Information:} Professors and students can see this
view. For the students, the information view will contain the description, the
instructions for completing the assignment, the due date, the student's score,
and the last submission date. The professor also has the ability to view the
scores of every student, and can also edit and delete information from the
assignment. A student can only view the information of the assignment and can
only view his own grade.

\item \textbf{Assignment Submissions:} Both students and professors can see the
assignment submissions. The difference is that the students can only see their
own submissions, whereas the professor can see the submissions of every single
student. the submission information will include the hash id, the student's
name, the submit date, the result of the submission, the score, the compile time
of the submission, and the run time of the submission. The professor will also
have an additional button that allows him to override the grade of a student.

\item \textbf{Assignment Test Cases:} Both students and professors can see the
assignment test cases, the difference is that the professor can add, edit and
delete test cases, whereas the student can only view them and develop his
assignment accordingly.

\item \textbf{Assignment Repositories:} Both the students and the professors can
see the repositories, but the student can only see his own repositories whereas
the professors can see the repositories of every student.

\item \textbf{Add test cases:} The professor can add test cases in the
assignment test cases view. This will ask for the input of the test case and the
expected output.

\item \textbf{Grade:} Both students and professor can see grades. The students
can see their own grade from the assignment view and from the course view. The
student will be able to view grade from specific assignments or total grades for
specific courses. The professors can see the grades of every student for his
courses and assignments. He can also modify their grades.

\item \textbf{Account Profile:} Both students and professors will be able to see
the account profiles.

\item \textbf{Account Security:} Both students and professors will be able to
see their account securities information. This will include the ability to
change their passwords and view and add ssh keys.


\end{enumerate}

\subsection{Back end}

The back end is the underlying server side software that manages the
communication and flow between the front end and the database server, the
repository manager, and the testing framework. The front end sends requests for
data to the back end using its REST API. The back end then proceeds with
establishing a connection to the MongoDB database server and then passing the
query's result to the front end. The database server returns query results with
JSON objects, and the back end does not have to serialize and deserialize the
information since the front end also uses JSON objects as discussed in the
earlier sections. The back end is also responsible of instantiating the sand
boxes where the submitted code and test cases will run. This is also invoked by
orders of the users when they interact with the front end of Panda Code Reviews.

\subsubsection{Node.js}

The back end service is being built in JavaScript using Node.js. Node.js is a
platform built on Google Chrome's Open Source V8 JavaScript runtime engine for
easily building fast and scalable network and server-side applications. It uses
JavaScript as its scripting language and achieves high throughput via an event
driven, non-blocking I/O model on a single-threaded event loop. It also contains
a built-in HTTP server library, which allows for easy deployment of a web server
without the need of external HTTP servers like Apache, Lighttpd or Nginx.
Node.js poses additional ease of use when it comes to interacting with MongoDB,
since it uses JSON-style documents as well. This means that no additional
library or wrapper is needed to communicate with the database.

\subsubsection{GitLab}

GitLab is open source software to help developers collaborate on code. It can be
used to create projects and repositories and manage access via HTTPS and SSH
authentication. The team found no other software that provides a similar amount
of API functions that GitLab provides. GitLab will be used as the repository
manager for Panda Code Reviews. The Node.js server will communicate with the
GitLab instance trough HTTP requests and will be able to manage repositories.
Management includes, basic CRUD (Create, Read, Update, Delete) of repositories
and users within.

\subsubsection{Endpoints}

An endpoint of a software designed with the service-oriented architecture (SOA)
pattern is the entry point of a particular service that the software provides,
where a service stands for a single activity. In the case of Panda Code Reviews,
the Endpoints are specified by the URLs that the client's browser must access in
order to receive the individual services from the back end that make up the web
application. The user need not know or specify these Endpoints, since the web
application's front end will already provide them to the client's browser. These
Endpoints are designed using the RESTful (representational state transfer) web
design architecture. The Endpoint URLs will be prefixed by api/ in order to
clearly separate the URLs that are intended to be part of the RESTful API of the
back end and the URLs that are intended to be accessed as part of the front end
application. The following tables specify the endpoints' method, URL and
description.

\setlength{\extrarowheight}{1.5pt}
    \begin{longtable}{|c|c|c|m{4cm}|}
 \caption{Users Endpoints\label{tab:addlabel}} \\
     \hline
    
    \centering  Method & Path & Query String & Description \\
    \hline \hline \endfirsthead
    
         \hline

  \centering  Method & Path & Query String & Description \\
    \hline \hline \endhead
    
    \endfoot  
    GET   & /users & ?course=\$cid & {Gets list of users} \\ \hline
    GET   & /users/\$uid &       & {Gets data from user \$uid} \\ \hline
    POST  & /users &       & {Creates users with JSON pyaload} \\ \hline
    PUT   & /users/\$uid &       & {Updates user \$uid data with JSON payload} \\ \hline
    DELETE & /users/\$uid &       & {Soft delete user \$uid} \\ \hline
    GET   & /users/\$uid/courses & ?year=\$y\&semester=\$s & {Gets courses of user \$uid} \\ \hline
    GET   & /users/\$id/assignments & ?course=\$cid & {Gets assigments of user \$uid} \\ \hline
    GET   & /users/\$id/submissions & ?assigment=\$aid\&course=\$cid & {Gets submissions of user \$uid} \\ \hline
\end{longtable}

\setlength{\extrarowheight}{1.5pt}
    \begin{longtable}{|c|c|c|m{4.5cm}|}
 \caption{Courses Endpoints\label{tab:addlabel}} \\
     \hline
    
    \centering  Method & Path & Query String & Description \\
    \hline \hline \endfirsthead
    
         \hline

    \centering  Method & Path & Query String & Description \\
    \hline \hline \endhead
    
    \endfoot 
    GET   & /courses & ?year=\$y\&semester=\$s & {Gets list of courses} \\ \hline
    GET   & /courses/\$cid &       & {Gets data from course \$cid} \\ \hline
    POST  & /courses &       & {Creates course with JSON payload} \\ \hline
    PUT   & /courses/\$cid &       & {Updates course \$cid data with JSON pyaload} \\ \hline
    DELETE & /courses/\$cid &       & {Soft delete course \$cid} \\ \hline
    GET   & /courses/\$cid/users & ?isGrade=\$bool & {Gets users of course \$cid} \\ \hline
    POST  & /courses/\$cid/users/\$uid &       & {Adds user \$uid to course \$cid} \\ \hline
    DELETE & /courses/\$cid/users/\$uid &       & {Removes user \$uid from course \$cid} \\ \hline
    GET   & /courses/\$cid/assignments &       & {Gets assigments of course \$cid} \\ \hline
    GET   & /courses/\$cid/submissions & ?assignment=\$aid & {Gets submissions of course \$cid} \\ \hline
\end{longtable}

\setlength{\extrarowheight}{1.5pt}
    \begin{longtable}{|c|c|c|c|}
 \caption{Assignments Endpoints\label{tab:addlabel}} \\
     \hline
    
    \centering  Method & Path & Description \\
    \hline \hline \endfirsthead
    
         \hline

    \centering  Method & Path & Description \\
    \hline \hline \endhead
    
    \endfoot 
    GET   & /assignments & {Gets a list of assignments} \\ \hline
    GET   & /assignments/\$aid & {Gets data from assignment \$aid} \\ \hline
    POST  & /assignments & {Creates assignment from JSON payload} \\ \hline
    PUT   & /assignments & {Updates assignment \$aid data with JSON payload} \\ \hline
    DELETE & /assignments/\$aid & {Soft delete assignment \$aid} \\ \hline
    POST  & /assignments/\$aid/test & {Creates a test case from assignment \$aid} \\ \hline
    DELETE & /assignments/\$aid/test/\$tid & {Deletes test case \$tid from assignment \$aid} \\ \hline
\end{longtable}

\setlength{\extrarowheight}{1.5pt}
    \begin{longtable}{|c|c|m{10cm}|}
 \caption{Submissions Endpoints\label{tab:addlabel}} \\
     \hline
    
    \centering  Method & Path & Description \\
    \hline \hline \endfirsthead
    
         \hline

    \centering  Method & Path & Description \\
    \hline \hline \endhead
    
    \endfoot 
    GET   & /submissions & {Gets a list of submissions} \\ \hline
    GET   & /submissions/\$sid & {Gets data from submissions \$sid} \\ \hline
    POST  & /submissions & {Creates submissions from JSON payload. Endpoint that queues code for evaluation} \\ \hline
\end{longtable}


\subsubsection{Codeworkers}
A codeworker will be a JavaScript object in the back end that will implement
methods for compiling and running the test cases of the assignment submission
that they are affiliated to. These workers are created when a students submits
an assignment for evaluation. The codeworkers are placed on a queue and they get
executed as they reach the front of such queue. The codeworkers then proceeds to
compile, execute and test the code provided to him trough the submission and he
then reports back the result.

Figure~\ref{fig:flow} contains a flow chart that depicts the flow of events of
the Codeworkers.

\begin{figure}[H]
	\centering
	\includegraphics[width=\textwidth]{img/flowchart}
	\caption{Codeworker flow chart\label{fig:flow}}
\end{figure}

\subsection{Database}

The database is used for storing all required information about the system and
its users.

\subsubsection{MongoDB}

MongoDB is a cross-platform document-oriented database system. It is classified
as a NoSQL database which provides a mechanism for storage retrieval that is not
as constrained as relational databases like MySQL or PostgreSQL. MongoDB
features JSON-style (JavaScript Object Notation) documents with dynamic schemas,
which allows simpler and quicker development. A dynamic schema approach enables
flexibility for quickly changing applications, and the JSON style documents
provide a unified language for passing database objects between the Node.js
back end and the front end, both written entirely in JavaScript. MongoDB is the
leading NoSQL DBMS, it is more stable than its closest competitor Redis.

\subsubsection{Entity-Relationship Diagram}

The collections have been defined following two different approaches: the
standard ER diagram that defines entities and the relationships with each other
and a JSON-like diagram that better models MongoDB's Documents and Collections.
Overall, we created structures for users (Administrators, Professors, Students
and TAs), courses, assignments and submissions.

\includepdf[pages={-}]{img/class_diagram.pdf}


\subsection{Deployment}

Continuous development has been set up in the deployment server collocated at
Amazon's EC2 infrastructure. This enables an always working version hosted in
the cloud which gets automatically deployed whenever a developer pushes code to
the master branch in Git. When the latest version is pushed, the Node Package
Manager (NPM) runs code tests to make sure that the latest push is healthy. If
it passes the tests, the server is shut down, any new dependencies are installed
with NPM, and the server is restarted with the new changes in effect. This all
happens automatically.

% This is the overview of the front end and back end architecture.
\includepdf[pages={-}]{img/tech_archi.pdf}
