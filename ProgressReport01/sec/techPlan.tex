\autsection{Technical Plan}{Nelián Colón, Samuel Rodríguez and Daniel Santiago}

\subsection{Web Application}
The web application is the whole system. It has many parts, which follow.

\subsection{Front end}

The frontend is being developed in HTML4 with some features of the new HTML5,
CSS3, and Javascript. The majority of the CSS is modified Bootstrap code, a
front end framework for faster and easier web development. JQuery and AngularJS
are being used as scripting frameworks for dynamic web pages.

\subsubsection{CSS}

The web application's layout is constructed on top of an HTML template that has
been created from scratch. Bootstrap containers and components are being used
for maintaining excellent responsiveness while resizing a browser window, as
well as for easily adopting the layout in mobile devices. Additionally,
Bootstrap includes many Javascript plugins for controlling some of the
component's animations. Some components have to be heavily modified for
suitability in the application.

Jetstrap, a Bootstrap interface builder, is being used for the majority of the
development of the front end. Sublime Text 2 is also being used for
development, and Git is the version control manager. Chrome Dev Tools and
Firebug plugin for Firefox are being used for debugging front end code.

\subsection{Javascript}

The dynamic elements of the CSS code is mostly being controlled by the
underlying Javascript which is being written with JQuery and AngularJS.

\subsubsection{Views}
% Parenthesis means that no mock exists yet.
Index

Login
(Sign Up)

Account Home

Course (Falta professor, create course | edit mode)

Assignment (falta edit mode, create assignment)
	Info
	Submissions
	Test Cases
	Repositories (Gitlab)

(Admin panel)
	(Schools)
	(Server status)

(Account Settings (Profile))
(Professor's side of grading)


\subsection{Back end}
The back end is the service that is drives the front end. The front end sends
requests for data to the back end, which proceeds with establishing a
connection to the MongoDB database server and then passing the query's result
to the front end. The back end is also responsible of instantiating the sand
boxes where the submitted code and test cases will run.

\subsubsection{Node.js}

Node.js is a platform built on Google Chrome's Open Source V8 JavaScript runtime
engine for easily building fast and scalable network and server-side
applications. It uses JavaScript as its scripting language and achieves high
throughput via an event driven, non-blocking I/O model on a single-threaded
event loop. It also contains a built-in HTTP server library, which allows for
easy deployment of a web server without the need of external HTTP servers like
Apache, Lighttpd or Nginx. Node.js poses additonal ease of use when it comes to
interacting with MongoDB, since it uses JSON-style documents as well. This means
that no additional library or wrapper is needed to communicate with the
database.

\subsubsection{GitLab}
GitLab is open source software to help developers collaborate on code. It can
be used to create projects and repositories and manage access.

\subsubsection{Endpoints}

\subsubsection{Codeworkers}

\subsection{Database}

The database is used for storing all required information about the system and
its users.

\subsubsection{MongoDB}

MongoDB is a cross-platform document-oriented database system. It is classified
as a NoSQL database which provides a mechanism for storage retrieval that is not
as constrained as relational databases like MySQL or PostgreSQL. MongoDB
features JSON-style (JavaScript Object Notation) documents with dynamic schemas,
which allows simpler and quicker development. A dynamic schema approach enables
flexibility for quickly changing applications, and the JSON style documents
provide a unified language for passing database objects between the Node.js
backend and the front end, both written entirely in Javascript. MongoDB is the
leading NoSQL DBMS, it is more stable than its closest competitor Redis. 

\subsubsection{Entity-Relationship Diagram}

We have defined the collections following two different approaches: the
standard ER diagram that defines entities and the relationships with each other
and a JSON-like diagram that better models MongoDB's Documents and Collections.
Overall, we created structures for users (Administrators, Professors, Students
and TAs), courses, assignments and submissions.


\subsection{Deployment}

We have set up continuous development in our deployment server colocated at
Amazon's EC2 infrastructure. This enables us to always have a working version
hosted in the cloud which gets automatically deployed whenever a developer
pushes code to the master branch in Git. When the latest version is pushed, the
Node Package Manager (NPM) runs code tests to make sure that the latest push is
healthy. If it passes the tests, the server is shut down, any new dependencies
are installed with NPM, and the server is restarted with the new changes in
effect. This all happens automatically.

%presents and describes design alternatives, and justifies all the choices made

%presents and describes system architecture

%present progress in the design with technical diagrams and description of system components (appendices required for calculations, and detailed documentation diagrams and descriptions)

%analyzes and justifies any departures with respect to original plan

%presents snapshots or other evidences

%check list:

%present system design overview
%present sys conceptual design (how it will look like)
%present software architecture
%present component description
%present user interface
%presents software progress assessment
%presents class diagrams
%presents software design justications
%presents communication interfaces