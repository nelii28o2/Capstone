\autsection{Design Criteria and Specifications}{Samuel Rodríguez and Daniel Santiago}


\begin{figure}[H]
	\centering
	\includegraphics[scale=0.15]{img/bigArquitectOverview}
	\caption{General overview of the system\label{arqu}}
\end{figure}

\subsection{Web Application}

Panda Code Reviews is a fully hosted web application. The web application
includes many parts, the main of these are the front end and the back end. The
front end is what the user sees and interacts with directly. It is very
important to design the front end with ease of use in mind, and almost equally
important is designing the interface with a clean and modern look. The back end
is the underlying software that manages the communication between the front end
interface and the database, repository manager and testing framework. The front
end and the back end are connected in various parts, so this must be properly
designed for ensuring compatibility and hassle-free communication. For example,
data passing between the database and the front end through the back end must be
concise, so a standard object model should be used. It is discussed in the
following sections the usage of JSON\cite{JSON} objects for this purpose.
Additionally, the URL Endpoints of the application must be specified for
compatibility between the front end and the back end. Figure~\ref{fig:archi}
displays a high level architectural diagram of the application. Following is the
description of each component.

\begin{figure}[H]
    \centering
    \includegraphics[width=\textwidth]{img/archi}
    \caption{Front end and Back end High Tech Diagram\label{fig:archi}}
\end{figure}

\subsection{Front end}

As mentioned earlier, it is very important to design the front end interface
with ease of use in mind, and it is also very important to design the interface
with a clean and modern look. The Model-View-Controller (MVC) pattern was
followed for developing the front end. The idea of the MVC pattern is to have a
good organization in the code between managing the application's data (model),
the application's logic (controller), and the application's presentation (view).
The view gets data from the model to display to the user. When a user interacts
with the application, the controller changes data in the model, and the data
gets passed from the model to the view so that it is displayed to the user. In
the case of Panda Code Reviews, the view of the application is the Document
Object Model (DOM) of the web pages. The controller of the application is
JavaScript code that uses AngularJS\cite{angular}, and the model and exchange
data format is JSON.

One of the most difficult tasks in developing the front end of the application
was ensuring good compatibility between the model, the view and the controller.
AngularJS makes this task easier by providing a templating engine in the client
side. It allows for automatic refreshing of data between the model, view and
controller. For example, if the controller of the application changes the model
data, AngularJS automatically refreshes the view so that it gets displayed to
the user. And if the view is changed by the user, AngularJS automatically
updates the model.

\subsubsection{View}

The view is handled by HTML and CSS. Panda Code Reviews uses
Bootstrap\cite{bootstrap}, an open source front end framework for faster and
easier development of the front end's view. It provides many CSS classes for
easily templating the web pages. It is also highly customizable, and provides a
JavaScript plugin interface for enabling animations and other features like
component pinning and affixes. These plugins might be activated by the
controller or by other means such as after a page loads in the client's browser.

\subsubsection{Controller}

The controller of the front end was developed in JavaScript. JavaScript is
a scripting language implemented by all major browsers and provides a very
extensive standard library for manipulating elements of the DOM and registering
event listeners. JQuery\cite{jquery}, a comprehensive JavaScript library, is
being used for simplifying and reusing many routines. Additionally, AngularJS is
also embedded in the controller. It is explained in the next section.

\subsubsection{Model and Communication with the Controller}

The model of the application is embedded in the controller and specified in the
database, which the front end has no direct access to. Instead, the front end
will communicate with the back end through a REST API, and the back end will
answer the front end's query with JSON objects. AngularJS will be used for
simplifying the communications between the front end's controller and model.
AngularJS will also be used for communicating with the back end through its REST
API. When the user interacts with the front end's view, AngularJS responds by
changing the model in the front end, which will then call the necessary
controller functions that will communicate with the back end through its REST
API. The back end will respond to the controller's query with JSON objects, and
AngularJS will manage the change of data between the controller, the model and
the view. AngularJS is also highly modifiable, so functionality can be
overridden so that it better suits Panda Code Reviews's functionalities.

\subsubsection{Views}

The team developed some static web pages with
HTML4/5 and Bootstrap CSS. This doubles as part of the implementation of the
front end and as mockup creation for the design phase. The application views
that have been planned by the team follows. Use case diagrams can be found in
Appendix~\ref{sec:useCase} and descriptions can be found in
Appendix~\ref{sec:useCases}. Screenshots of what the users is expected to see
are found in Appendix~\ref{sec:mockups}. Sequential diagrams that describe the
actions that the users can perform can be found in Appendix~\ref{sec:seqs}. The
following list enumerates the views and briefly describe its content and
available actions.

\subsection{Back end}

The back end is the underlying server side software that manages the
communication and flow between the front end and the database server, the
repository manager, and the testing framework. The front end sends requests for
data to the back end using its REST API. The back end then proceeds with
establishing a connection to the MongoDB\cite{mongodb} database server and then
passing the query's result to the front end. The database server returns query
results with JSON objects, and the back end does not have to serialize and
deserialize the information since the front end also uses JSON objects as
discussed in the earlier sections. The back end is also responsible of
instantiating the sand boxes where the submitted code and test cases will run.
This is also invoked by orders of the users when they interact with the front
end of Panda Code Reviews.

\subsubsection{Node.js}

The back end service is being built in JavaScript using Node.js\cite{nodejs}.
Node.js is a platform built on Google Chrome's Open Source V8 JavaScript runtime
engine for easily building fast and scalable network and server-side
applications. It uses JavaScript as its scripting language and achieves high
throughput via an event driven, non-blocking I/O model on a single-threaded
event loop. It also contains a built-in HTTP server library, which allows for
easy deployment of a web server without the need of external HTTP servers like
Apache, Lighttpd or Nginx. Node.js poses additional ease of use when it comes to
interacting with MongoDB, since it uses JSON-style documents as well. This means
that no additional library or wrapper is needed to communicate with the
database.

\subsubsection{GitLab}

GitLab\cite{gitlab} is open source software to help developers collaborate on
code. It can be used to create projects and repositories and manage access via
HTTPS and SSH authentication. The team found no other software that provides a
similar amount of API functions that GitLab provides. GitLab will be used as the
repository manager for Panda Code Reviews. The Node.js server will communicate
with the GitLab instance trough HTTP requests and will be able to manage
repositories. Management includes, basic CRUD (Create, Read, Update, Delete) of
repositories and users within.

\subsubsection{Endpoints}

An endpoint of a software designed with the service-oriented architecture (SOA)
pattern is the entry point of a particular service that the software provides,
where a service stands for a single activity. In the case of Panda Code Reviews,
the Endpoints are specified by the URLs that the client's browser must access in
order to receive the individual services from the back end that make up the web
application. The user need not know or specify these Endpoints, since the web
application's front end will already provide them to the client's browser. These
Endpoints are designed using the RESTful (representational state transfer) web
design architecture. The Endpoint URLs will be prefixed by api/ in order to
clearly separate the URLs that are intended to be part of the RESTful API of the
back end and the URLs that are intended to be accessed as part of the front end
application. The following tables specify the endpoints' method, URL and
description.

\setlength{\extrarowheight}{1.5pt}
    \begin{longtable}{|c|c|c|m{4cm}|}
 \caption{Users Endpoints\label{tab:usersends}} \\
     \hline
    
    \centering  Method & Path & Query String & Description \\
    \hline \hline \endfirsthead
    
         \hline

  \centering  Method & Path & Query String & Description \\
    \hline \hline \endhead
    
    \endfoot  
    GET   & /users & ?course=\$cid & {Gets list of users} \\ \hline
    GET   & /users/\$uid &       & {Gets data from user \$uid} \\ \hline
    POST  & /users &       & {Creates users with JSON payload} \\ \hline
    PUT   & /users/\$uid &       & {Updates user \$uid data with JSON payload} \\ \hline
    DELETE & /users/\$uid &       & {Soft delete user \$uid} \\ \hline
    GET   & /users/\$uid/courses & ?year=\$y\&semester=\$s & {Gets courses of user \$uid} \\ \hline
    GET   & /users/\$id/assignments & ?course=\$cid & {Gets assigments of user \$uid} \\ \hline
    GET   & /users/\$id/submissions & ?assigment=\$aid\&course=\$cid & {Gets submissions of user \$uid} \\ \hline
\end{longtable}

\setlength{\extrarowheight}{1.5pt}
    \begin{longtable}{|c|c|c|m{4.5cm}|}
 \caption{Courses Endpoints\label{tab:coursesends}} \\
     \hline
    
    \centering  Method & Path & Query String & Description \\
    \hline \hline \endfirsthead
    
         \hline

    \centering  Method & Path & Query String & Description \\
    \hline \hline \endhead
    
    \endfoot 
    GET   & /courses & ?year=\$y\&semester=\$s & {Gets list of courses} \\ \hline
    GET   & /courses/\$cid &       & {Gets data from course \$cid} \\ \hline
    POST  & /courses &       & {Creates course with JSON payload} \\ \hline
    PUT   & /courses/\$cid &       & {Updates course \$cid data with JSON pyaload} \\ \hline
    DELETE & /courses/\$cid &       & {Soft delete course \$cid} \\ \hline
    GET   & /courses/\$cid/users & ?isGrade=\$bool & {Gets users of course \$cid} \\ \hline
    POST  & /courses/\$cid/users/\$uid &       & {Adds user \$uid to course \$cid} \\ \hline
    DELETE & /courses/\$cid/users/\$uid &       & {Removes user \$uid from course \$cid} \\ \hline
    GET   & /courses/\$cid/assignments &       & {Gets assigments of course \$cid} \\ \hline
    GET   & /courses/\$cid/submissions & ?assignment=\$aid & {Gets submissions of course \$cid} \\ \hline
\end{longtable}

\setlength{\extrarowheight}{1.5pt}
    \begin{longtable}{|c|c|c|c|}
 \caption{Assignments Endpoints\label{tab:assignmentsends}} \\
     \hline
    
    \centering  Method & Path & Description \\
    \hline \hline \endfirsthead
    
         \hline

    \centering  Method & Path & Description \\
    \hline \hline \endhead
    
    \endfoot 
    GET   & /assignments & {Gets a list of assignments} \\ \hline
    GET   & /assignments/\$aid & {Gets data from assignment \$aid} \\ \hline
    POST  & /assignments & {Creates assignment from JSON payload} \\ \hline
    PUT   & /assignments & {Updates assignment \$aid data with JSON payload} \\ \hline
    DELETE & /assignments/\$aid & {Soft delete assignment \$aid} \\ \hline
    POST  & /assignments/\$aid/test & {Creates a test case from assignment \$aid} \\ \hline
    DELETE & /assignments/\$aid/test/\$tid & {Deletes test case \$tid from assignment \$aid} \\ \hline
\end{longtable}

\setlength{\extrarowheight}{1.5pt}
    \begin{longtable}{|c|c|m{10cm}|}
 \caption{Submissions Endpoints\label{tab:submissionsends}} \\
     \hline
    
    \centering  Method & Path & Description \\
    \hline \hline \endfirsthead
    
         \hline

    \centering  Method & Path & Description \\
    \hline \hline \endhead
    
    \endfoot 
    GET   & /submissions & {Gets a list of submissions} \\ \hline
    GET   & /submissions/\$sid & {Gets data from submissions \$sid} \\ \hline
    POST  & /submissions & {Creates submissions from JSON payload. Endpoint that queues code for evaluation} \\ \hline
\end{longtable}


\subsubsection{Codeworkers}

A codeworker will be a JavaScript object in the back end that will implement
methods for compiling and running the test cases of the assignment submission
that they are affiliated to. These workers are created when a student submits an
assignment for evaluation trough the front end. The front end proceeds by
passing the information to the back end server trough the submission endpoint.
The codeworkers are placed on a queue and they get executed as they reach the
front of the queue. The codeworkers then proceed to compile, execute and test
the code provided to it through the submission and then reports back the
result.

Figure~\ref{fig:flow} contains a flow chart that depicts the flow of events of
the codeworkers.

\begin{figure}[H]
	\centering
	\includegraphics[width=\textwidth]{img/flowchart}
	\caption{Codeworker flow chart\label{fig:flow}}
\end{figure}

\subsubsection{Email Server}

Postfix\cite{postfix} will be used as the mail transfer agent that delivers and
receives email for the @pandacode.org domain. Emails are sent for account
verification and password reset mechanism. It is also used for support and
contacting emails such as help@pandacode.org and market@pandacode.org


\subsection{Database}

The database is used for storing all required information about the system and
its users.

\subsubsection{MongoDB}

MongoDB\cite{mongodb} is a cross-platform document-oriented database system. It
is classified as a NoSQL database which provides a mechanism for storage
retrieval that is not as constrained as relational databases like MySQL or
PostgreSQL. MongoDB features JSON-style (JavaScript Object Notation) documents
with dynamic schemas, which allows simpler and quicker development. A dynamic
schema approach enables flexibility for quickly changing applications, and the
JSON style documents provide a unified language for passing database objects
between the Node.js back end and the front end, both written entirely in
JavaScript. MongoDB is the leading NoSQL DBMS, it is more stable than its
closest competitor Redis.

\subsubsection{Entity-Relationship Diagram}

The collections and documents of the database have been defined following two
different approaches: the standard ER diagram that defines entities and the
relationships with each other shown in Figure~\ref{fig:classdiagram} and a JSON-
like diagram that better models MongoDB's Documents and Collections shown in
Figure~\ref{fig:jsondiagram}. Overall, we created structures for users
(Administrators, Professors, Students and TAs), courses, assignments and
submissions.

\begin{figure}[H]
    \centering
    \includegraphics[width=\textwidth]{img/class-diagram}
    \caption{Class Diagram\label{fig:classdiagram}}
\end{figure}

\begin{figure}[H]
    \centering
    \includegraphics[width=\textwidth]{img/json-diagram}
    \caption{JSON Diagram\label{fig:jsondiagram}}
\end{figure}


\subsection{Deployment}

Continuous development has been set up in the deployment server collocated at
Amazon's EC2 infrastructure. This enables an always working version hosted in
the cloud which gets automatically deployed whenever a developer pushes code to
the master branch in Git. When the latest version is pushed, the Node Package
Manager (NPM) runs code tests to make sure that the latest push is healthy. If
it passes the tests, the server is shut down, any new dependencies are installed
with NPM, and the server is restarted with the new changes in effect. This all
happens automatically.
