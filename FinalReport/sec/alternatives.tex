\autsection{Alternatives}{Daniel Santiago}
\label{sec:alternatives}
There are many available technologies that Team Aguacate could have used instead of the
selected ones that were described in Section~\ref{sec:design}.

\subsection{Front End}
In the front end, it is inevitable to use HTML and CSS for the views. This is the only
markup technology that is supported by all modern browsers. JavaScript is also the only
scripting language that is supported by all modern browsers, therefore, it is inevitable
to use JavaScript for all Front End scripting. The team could have used a similar technology
to replace AngularJS. Some other frameworks exist, but the only viable alternative would
have been EmberJS. EmberJS is a framework for creating web applications just like AngularJS.
The main difference between EmberJS and AngularJS is that AngularJS automatically syncs all
model objects with the view and vice-versa. This makes for an easier development pattern.
Thus, AngularJS was chosen as the front end JavaScript framework.

\subsection{Back End}
Node.js and MongoDB were the two main technologies used for the back end. There are many
alternatives that could have been used. For the database, an SQL DBMS could have been used,
for example the open sourced MySQL or PostgreSQL DBMS. The main reason why MongoDB was used
is because MongoDB's schema is dynamic. There is no need to initially specify a schema. This
allowed Team Aguacate to change the schema on the fly while developing the application.
The other plus side of using MongoDB for the database management system is that the objects
that the database uses for storing information are represented as JSON (Javascript Object
Notation). This is a major convenience, since Node.js and AngularJS both use JSON for
representing data internally. If an SQL database was used, a database driver and middleware
would have been needed, which greatly complicates the development procedure. Some alternatives to Node.js include HTTP servers like Apache and Lighttpd. There are many
disadvantages of using these servers. One of them is that they do not support JavaScript nor
JSON, which results in a lot of duplicated code. Furthermore, they do not provide
development libraries. Also, the servers do not provide a package management system, which
makes plugin installation a problem that could be easily solved by just using Node.js.
