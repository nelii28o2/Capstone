\part*{Executive Summary}

Assignments for Software Engineering classes have particular characteristics
that make them time-consuming to grade by professors and teaching assistants
(TA). In the majority of cases, they are handed in electronically. The grader
then runs the code with some specific input to verify its correctness. Students
might choose to use different programming languages, different IDEs, and
different tests, making grading even more difficult. Additionally, many students
write code in different styles, which may make it harder to read for a person
that has been accustomed to a particular coding style. Furthermore,
communication between the grader and the student may be slow, often relying in
sluggish e-mail exchanges. All of this makes grading software assignments an
enduring and tedious task.

Panda Code Reviews has been modularly designed following industry leading
standards and using the newest technology for web applications development. It
is being built to be a fully hosted web application in the cloud that will
eliminate many of the difficulties that come with grading software assignments.
The application will host repositories that contain students' code. When a
student submits code for an assignment, the system notifies the grader and
automatically runs test cases previously specified by the grader. The results of
the test cases are sent to the grader. The grader also has the option to add
line-by-line comments to the student's code, and run the application's linter
tools to verify code style. Finally, the grader will have the option to assign a
grade to the student.

The team has been able to progress on the development of Panda Code Reviews very
closely to what was expected. The design of the system has been finished, and
implementation started earlier than what was originally planned. This is because
instead of creating mockups for design, html and css pages were built. This
allows the usage of mockup code as production code, which means that mockup
production is actually the beginning of the implementation of the front end.
Additionally, a small prototype for compiling and running Java code that is
uploaded via the web interface to the server has been built. This is part of the
back end implementation. Thus, the team has begun early in these two aspects of
the implementation of Panda Code Reviews.

There have also been some small difficulties during design and the early phase
of implementation. For repository management, an open source software called
GitLab will be installed on the production server. The software needs at least
768MB of RAM, which was more than what was available in the Amazon EC2
development instance. This has been fixed by spawning a larger M1 Amazon EC2
Instance. Also, The team has not been able to incorporate Google Ads from
Adsense because a domain name with content is needed for creation of the Adsense
account. Therefore, the setup of ads has been postponed. Even without these
small difficulties, much work is still needed for the completion of the project.
Notwithstanding it is going very similar as initially planned. The changes are
shown in the accompanying Gantt Chart.

Up to this date, \$11.34 have been used for the Amazon EC2 Micro instance that
contains the development machine and the Amazon EC2 Medium instance that
includes the installed GitLab software. The team has not had any other hardware
or software costs up to the time of this writing. Thus, the projected
expenditure of the project is still the same as what was initially proposed, and
the budget costs have been as initially expected.
