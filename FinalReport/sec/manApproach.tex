\autsection{Methods and Approach to the solution}{Nelián Colón}

At first, the team developed different parts of
the system as they were all loosely coupled. Test driven development was
done, meaning that after each part was successfully developed and unit tested,
only then they were integrated and tested from a bigger perspective. As shown in Figure~\ref{arqu}, this project
was divided into 2 major parts, these are: front end client, which contains everything that is seen by the user, and back end server, which is responsible for managing all the data and communications within the modules. The team used a single cloud hosted Git repository (GitHub) to host the project's
code. Moreover, the
team followed the Scrum agile development process.


\subsection{Team Management}

Meetings were held at the start and end of each sprint, these meeting were
sprint planning and sprint retrospective. Each task had a leader and an
assistant. The leader of each task was completely responsible for the task and
the assistant helped whenever the leader got stuck on a problem. Tasks had a weight that was decided during the sprint planning
meeting. Table~\ref{tasks} contains the main tasks distribution for the project.
For a more detailed list of task distribution see the Gantt Chart included as a
separate file.

\setlength{\extrarowheight}{1.5pt}
  \begin{longtable}{|c|c|c|c|}
 \caption{Task Distribution \label{tasks}} \\
   \hline
  
  \centering Task Title & Nelián & Samuel & Daniel \\
  \hline \hline \endfirsthead
  
     \hline

	\centering Task Title & Nelián & Samuel & Daniel \\  
	\hline \hline \endhead
  
  \endfoot  
  
  Project Management and Team Organization & Lead & Assist & \\ \hline
  Web Front end & & Lead & Assist \\ \hline
  Back end Server & & Assist & Lead \\ \hline
  Accounts \& Repositories & Lead & Assist &\\ \hline
   \end{longtable}
   
\subsection{Project Changes}
There was only a minor change in the originally proposed objectives on the project. A change request form is included with this document in a separate file.
   
\subsection{Testing and Quality Control Procedures}

To ensure the quality of the software product, the team employed both static
and dynamic testing. Static testing was performed via code reviews for every
single Git repository commit, performed by the other two team members
that did not submit the comment. The review included running the change in each of the team
member's work station, and visually inspecting the code. This helped detect
bugs and also maintain good coding guidelines and style. Dynamic testing was
performed via different testing levels. The first level was unit testing,
which was required for every commit in which unit tests were applicable. The next level of testing was integration testing, which occured even before integration begins. The last testing level was system
testing, which occured on Amazon's infracture on every push of the
repository. Continuous deployment was used to ensure
that the system was working everytime, and the system tests ran after the
change was built remotely.

Copies of the test plans, test procedures and results are included with this document in a separate file.

\subsection{Schedule}

Only minor changes were done to the original schedule. These changes were the ones done by the professors to the class schedule. This only translated to more time to do some tasks and hence compensated possible delays.